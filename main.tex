\input{./src/main.sty}

\begin{document}

\input{./src/titlepage.tex}

\pagebreak

\begin{enumerate}
  \item Using the regular solution model, determine the phase diagram
    of an isomorphous alloy that has
    been determined to have the solution parameters shown below. Use
    a computer (e.g., Python) to plot
    the free energy curves for strategic temperatures. The maximum
    temperature required for plotting
    your G-X curves should be easy to figure out. Figuring out what
    other strategic temperatures to use
    will require either playing around with the plots with respect to
    temperature (easy but potentially
    tedious) or using your thermodynamic prowess to directly solve
    for a useful intermediate temperature
    (harder and potentially much faster, but not required on this
    assignment). Then, produce at least
    eight different free energy plots at strategic temperatures. For
    this assignment, you can graphically
    determine the phase boundaries on the eight different plots and
    fill in the rest by hand to produce the
    phase diagram (i.e., connect the dots).

    \begin{align*}
      T_1^{\alpha\to L} &= \SI{1550}{\kelvin} \\
      T_2^{\alpha\to L} &= \SI{1200}{\kelvin} \\
      \Delta S_1^{0,\alpha\to L} &= \SI{8}{\joule\per\mole\per\kelvin} \\
      \Delta S_2^{0,\alpha\to L} &= \SI{11}{\joule\per\mole\per\kelvin} \\
      \omega_0^\alpha &= \SI{19500}{\joule\per\mole} \\
      \omega_0^L &= \SI{1500}{\joule\per\mole}
    \end{align*}

    \boxedanswer{
      Use the solid $\alpha$ phase as a reference. The regular
      solution model gives the following (for the solid phase):

      \begin{equation*}
        \Delta G_{\text{mix}}^{(i)}(x,T) = \omega^{(i)}x(1-x) +
        RT[x\ln x + (1 - x)\ln (1-x)]
      \end{equation*}

      For the liquid phase, apply an offset:

      \begin{align*}
        \Delta G^L(x,T) &= (1 - X)\Delta G_1^\circ(T) + x\Delta
        G_2^\circ(T) + \Delta G_{\text{mix}}^{(\alpha)}(x,T) \\
        &
        \begin{aligned}
          \Delta H_f &= \Delta S^\circ T_m \\
          \Delta G^\circ &=\Delta S^\circ T_m - \Delta S^\circ T \\
          \Delta G_1^\circ(T) &= \Delta S_1^0(T_{m,1} - T) \\
          \Delta G_2^\circ(T) &= \Delta S_2^0(T_{m,2} - T) \\
        \end{aligned} \\
      \end{align*}
    }

    \pagebreak

  \item The free energy of mixing of a liquid solution is accurately modeled by:

    \begin{equation*}
      \Delta G_{\text{mix}}\{L; L\}=8400X_AX_B+RT(X_A\ln X_A+XB\ln X_B)
    \end{equation*}

    \begin{enumerate}[(a)]
      \item Compute and plot (i.e., using a computer) the activity of
        component B ($a_B$) as a function of
        composition at \SI{600}{\kelvin} in this solution.

        \boxedanswer{

          \begin{align*}
            \Delta \overline G_B &= \Delta G_{\text{mix}} + (1 - X_B)
            \frac{d \Delta G_{\text{mix}}}{dX_B} \tag{8.28} \\
            \Delta \overline G_B &= RT\ln a_B \\
            a_B &= \exp\left(\frac{\Delta G_{\text{mix}} + (1 - X_B)
            \frac{d \Delta G_{\text{mix}}}{dX_B}}{RT}\right)
          \end{align*}

          \centering
          \includegraphics[width=0.6\textwidth]{./assets/fig_1.png}
        }

        \pagebreak

      \item Now, plot the same thing, but assume the reference states in
        the plot are $\{L; \alpha\}$ and pure B has a
        free energy of fusion of \SI{-1200}{\joule\per\mole} at
        \SI{600}{\kelvin}.

        \boxedanswer{

          To account for the change in reference, the
          free energy in the exponential term above must be
          augmented by adding in the free energy of fusion.

          \begin{align*}
            \mu_B - \mu_B^{(\alpha)} &= (\mu_B -
            \mu_B^{\text{(liquid)}}) + \Delta G_{\text{fus}} \\
            a_B^{(\alpha)} &=
            a_B^{\text{(liquid)}}\exp\left(\frac{\Delta
            G_{\text{fus}}}{RT}\right)
          \end{align*}

          \centering
          \includegraphics[width=0.6\textwidth]{./assets/fig_2.png}

        }

    \end{enumerate}

    \pagebreak

  \item Given the following equation for excess free energy of mixing
    of a solution, determine the listed
    changes in properties due to mixing \SI{35}{\mole} of A with
    \SI{65}{\mole}
    of B at \SI{298}{\kelvin} and atmospheric
    pressure. I suggest you work symbolically until calculating the
    final answer of each part:

    \begin{align*}
      \Delta G_{\text{mix}}^{\text{xs}} &= a(1 - bT)(1 - cP)X_AX_B \\
      a &= \SI{10500}{\joule\per\mole} \\
      b &= \SI{3e-4}{\per\kelvin} \\
      c &= \SI{8e-10}{\per\pascal}
    \end{align*}

    \begin{enumerate}[(a)]
      \item  Total Gibbs free energy change of mixing.

        \boxedanswer{
          Put answer here.
        }
      \item  Partial molar Gibbs free energy change of component A.

        \boxedanswer{
          Put answer here.
        }
      \item  Partial molar Gibbs free energy change of component B.

        \boxedanswer{
          Put answer here.
        }
      \item  Total entropy change of mixing.

        \boxedanswer{
          Put answer here.
        }
      \item  Total enthalpy change of mixing.

        \boxedanswer{
          Put answer here.
        }
      \item  Is this a regular solution? Concisely justify your answer.

        \boxedanswer{
          Put answer here.
        }
    \end{enumerate}

\end{enumerate}

\pagebreak

\section*{Supporting code:}
\inputminted{julia}{./calculations/src/calculations.jl}

\end{document}


\begin{document}

\input{./src/titlepage.tex}

\pagebreak

\begin{enumerate}
  \item Using the regular solution model, determine the phase diagram
    of an isomorphous alloy that has
    been determined to have the solution parameters shown below. Use
    a computer (e.g., Python) to plot
    the free energy curves for strategic temperatures. The maximum
    temperature required for plotting
    your G-X curves should be easy to figure out. Figuring out what
    other strategic temperatures to use
    will require either playing around with the plots with respect to
    temperature (easy but potentially
    tedious) or using your thermodynamic prowess to directly solve
    for a useful intermediate temperature
    (harder and potentially much faster, but not required on this
    assignment). Then, produce at least
    eight different free energy plots at strategic temperatures. For
    this assignment, you can graphically
    determine the phase boundaries on the eight different plots and
    fill in the rest by hand to produce the
    phase diagram (i.e., connect the dots).

    \begin{align*}
      T_1^{\alpha\to L} &= \SI{1550}{\kelvin} \\
      T_2^{\alpha\to L} &= \SI{1200}{\kelvin} \\
      \Delta S_1^{0,\alpha\to L} &= \SI{8}{\joule\per\mole\per\kelvin} \\
      \Delta S_2^{0,\alpha\to L} &= \SI{11}{\joule\per\mole\per\kelvin} \\
      \omega_0^\alpha &= \SI{19500}{\joule\per\mole} \\
      \omega_0^L &= \SI{1500}{\joule\per\mole}
    \end{align*}

    \boxedanswer{
      Use the solid $\alpha$ phase as a reference. The regular
      solution model gives the following (for the solid phase):

      \begin{equation*}
        \Delta G_{\text{mix}}^{(i)}(x,T) = \omega^{(i)}x(1-x) +
        RT[x\ln x + (1 - x)\ln (1-x)]
      \end{equation*}

      For the liquid phase, apply an offset:

      \begin{align*}
        \Delta G^L(x,T) &= (1 - X)\Delta G_1^\circ(T) + x\Delta
        G_2^\circ(T) + \Delta G_{\text{mix}}^{(\alpha)}(x,T) \\
        &
        \begin{aligned}
          \Delta H_f &= \Delta S^\circ T_m \\
          \Delta G^\circ &=\Delta S^\circ T_m - \Delta S^\circ T \\
          \Delta G_1^\circ(T) &= \Delta S_1^0(T_{m,1} - T) \\
          \Delta G_2^\circ(T) &= \Delta S_2^0(T_{m,2} - T) \\
        \end{aligned} \\
      \end{align*}
    }

    \pagebreak

  \item The free energy of mixing of a liquid solution is accurately modeled by:

    \begin{equation*}
      \Delta G_{\text{mix}}\{L; L\}=8400X_AX_B+RT(X_A\ln X_A+XB\ln X_B)
    \end{equation*}

    \begin{enumerate}[(a)]
      \item Compute and plot (i.e., using a computer) the activity of
        component B ($a_B$) as a function of
        composition at \SI{600}{\kelvin} in this solution.

        \boxedanswer{

          \begin{align*}
            \Delta \overline G_B &= \Delta G_{\text{mix}} + (1 - X_B)
            \frac{d \Delta G_{\text{mix}}}{dX_B} \tag{8.28} \\
            \Delta \overline G_B &= RT\ln a_B \\
            a_B &= \exp\left(\frac{\Delta G_{\text{mix}} + (1 - X_B)
            \frac{d \Delta G_{\text{mix}}}{dX_B}}{RT}\right)
          \end{align*}

          \centering
          \includegraphics[width=0.6\textwidth]{./assets/fig_1.png}
        }

        \pagebreak

      \item Now, plot the same thing, but assume the reference states in
        the plot are $\{L; \alpha\}$ and pure B has a
        free energy of fusion of \SI{-1200}{\joule\per\mole} at
        \SI{600}{\kelvin}.

        \boxedanswer{

          To account for the change in reference, the
          free energy in the exponential term above must be
          augmented by adding in the free energy of fusion.

          \begin{align*}
            \mu_B - \mu_B^{(\alpha)} &= (\mu_B -
            \mu_B^{\text{(liquid)}}) + \Delta G_{\text{fus}} \\
            a_B^{(\alpha)} &=
            a_B^{\text{(liquid)}}\exp\left(\frac{\Delta
            G_{\text{fus}}}{RT}\right)
          \end{align*}

          \centering
          \includegraphics[width=0.6\textwidth]{./assets/fig_2.png}

        }

    \end{enumerate}

    \pagebreak

  \item Given the following equation for excess free energy of mixing
    of a solution, determine the listed
    changes in properties due to mixing \SI{35}{\mole} of A with
    \SI{65}{\mole}
    of B at \SI{298}{\kelvin} and atmospheric
    pressure. I suggest you work symbolically until calculating the
    final answer of each part:

    \begin{align*}
      \Delta G_{\text{mix}}^{\text{xs}} &= a(1 - bT)(1 - cP)X_AX_B \\
      a &= \SI{10500}{\joule\per\mole} \\
      b &= \SI{3e-4}{\per\kelvin} \\
      c &= \SI{8e-10}{\per\pascal}
    \end{align*}

    \begin{enumerate}[(a)]
      \item  Total Gibbs free energy change of mixing.

        \boxedanswer{
          Put answer here.
        }
      \item  Partial molar Gibbs free energy change of component A.

        \boxedanswer{
          Put answer here.
        }
      \item  Partial molar Gibbs free energy change of component B.

        \boxedanswer{
          Put answer here.
        }
      \item  Total entropy change of mixing.

        \boxedanswer{
          Put answer here.
        }
      \item  Total enthalpy change of mixing.

        \boxedanswer{
          Put answer here.
        }
      \item  Is this a regular solution? Concisely justify your answer.

        \boxedanswer{
          Put answer here.
        }
    \end{enumerate}

\end{enumerate}

\pagebreak

\section*{Supporting code:}
\inputminted{julia}{./calculations/src/calculations.jl}

\end{document}


\begin{document}

\input{./src/titlepage.tex}

\pagebreak

\begin{enumerate}
  \item Using the regular solution model, determine the phase diagram
    of an isomorphous alloy that has
    been determined to have the solution parameters shown below. Use
    a computer (e.g., Python) to plot
    the free energy curves for strategic temperatures. The maximum
    temperature required for plotting
    your G-X curves should be easy to figure out. Figuring out what
    other strategic temperatures to use
    will require either playing around with the plots with respect to
    temperature (easy but potentially
    tedious) or using your thermodynamic prowess to directly solve
    for a useful intermediate temperature
    (harder and potentially much faster, but not required on this
    assignment). Then, produce at least
    eight different free energy plots at strategic temperatures. For
    this assignment, you can graphically
    determine the phase boundaries on the eight different plots and
    fill in the rest by hand to produce the
    phase diagram (i.e., connect the dots).

    \begin{align*}
      T_1^{\alpha\to L} &= \SI{1550}{\kelvin} \\
      T_2^{\alpha\to L} &= \SI{1200}{\kelvin} \\
      \Delta S_1^{0,\alpha\to L} &= \SI{8}{\joule\per\mole\per\kelvin} \\
      \Delta S_2^{0,\alpha\to L} &= \SI{11}{\joule\per\mole\per\kelvin} \\
      \omega_0^\alpha &= \SI{19500}{\joule\per\mole} \\
      \omega_0^L &= \SI{1500}{\joule\per\mole}
    \end{align*}

    \boxedanswer{
      Use the solid $\alpha$ phase as a reference. The regular
      solution model gives the following (for the solid phase):

      \begin{equation*}
        \Delta G_{\text{mix}}^{(i)}(x,T) = \omega^{(i)}x(1-x) +
        RT[x\ln x + (1 - x)\ln (1-x)]
      \end{equation*}

      For the liquid phase, apply an offset:

      \begin{align*}
        \Delta G^L(x,T) &= (1 - X)\Delta G_1^\circ(T) + x\Delta
        G_2^\circ(T) + \Delta G_{\text{mix}}^{(\alpha)}(x,T) \\
        &
        \begin{aligned}
          \Delta H_f &= \Delta S^\circ T_m \\
          \Delta G^\circ &=\Delta S^\circ T_m - \Delta S^\circ T \\
          \Delta G_1^\circ(T) &= \Delta S_1^0(T_{m,1} - T) \\
          \Delta G_2^\circ(T) &= \Delta S_2^0(T_{m,2} - T) \\
        \end{aligned} \\
      \end{align*}
    }

    \pagebreak

  \item The free energy of mixing of a liquid solution is accurately modeled by:

    \begin{equation*}
      \Delta G_{\text{mix}}\{L; L\}=8400X_AX_B+RT(X_A\ln X_A+XB\ln X_B)
    \end{equation*}

    \begin{enumerate}[(a)]
      \item Compute and plot (i.e., using a computer) the activity of
        component B ($a_B$) as a function of
        composition at \SI{600}{\kelvin} in this solution.

        \boxedanswer{

          \begin{align*}
            \Delta \overline G_B &= \Delta G_{\text{mix}} + (1 - X_B)
            \frac{d \Delta G_{\text{mix}}}{dX_B} \tag{8.28} \\
            \Delta \overline G_B &= RT\ln a_B \\
            a_B &= \exp\left(\frac{\Delta G_{\text{mix}} + (1 - X_B)
            \frac{d \Delta G_{\text{mix}}}{dX_B}}{RT}\right)
          \end{align*}

          \centering
          \includegraphics[width=0.6\textwidth]{./assets/fig_1.png}
        }

        \pagebreak

      \item Now, plot the same thing, but assume the reference states in
        the plot are $\{L; \alpha\}$ and pure B has a
        free energy of fusion of \SI{-1200}{\joule\per\mole} at
        \SI{600}{\kelvin}.

        \boxedanswer{

          To account for the change in reference, the
          free energy in the exponential term above must be
          augmented by adding in the free energy of fusion.

          \begin{align*}
            \mu_B - \mu_B^{(\alpha)} &= (\mu_B -
            \mu_B^{\text{(liquid)}}) + \Delta G_{\text{fus}} \\
            a_B^{(\alpha)} &=
            a_B^{\text{(liquid)}}\exp\left(\frac{\Delta
            G_{\text{fus}}}{RT}\right)
          \end{align*}

          \centering
          \includegraphics[width=0.6\textwidth]{./assets/fig_2.png}

        }

    \end{enumerate}

    \pagebreak

  \item Given the following equation for excess free energy of mixing
    of a solution, determine the listed
    changes in properties due to mixing \SI{35}{\mole} of A with
    \SI{65}{\mole}
    of B at \SI{298}{\kelvin} and atmospheric
    pressure. I suggest you work symbolically until calculating the
    final answer of each part:

    \begin{align*}
      \Delta G_{\text{mix}}^{\text{xs}} &= a(1 - bT)(1 - cP)X_AX_B \\
      a &= \SI{10500}{\joule\per\mole} \\
      b &= \SI{3e-4}{\per\kelvin} \\
      c &= \SI{8e-10}{\per\pascal}
    \end{align*}

    \begin{enumerate}[(a)]
      \item  Total Gibbs free energy change of mixing.

        \boxedanswer{
          Put answer here.
        }
      \item  Partial molar Gibbs free energy change of component A.

        \boxedanswer{
          Put answer here.
        }
      \item  Partial molar Gibbs free energy change of component B.

        \boxedanswer{
          Put answer here.
        }
      \item  Total entropy change of mixing.

        \boxedanswer{
          Put answer here.
        }
      \item  Total enthalpy change of mixing.

        \boxedanswer{
          Put answer here.
        }
      \item  Is this a regular solution? Concisely justify your answer.

        \boxedanswer{
          Put answer here.
        }
    \end{enumerate}

\end{enumerate}

\pagebreak

\section*{Supporting code:}
\inputminted{julia}{./calculations/src/calculations.jl}

\end{document}


\begin{document}

\input{./src/titlepage.tex}

\pagebreak

\begin{enumerate}
  \item Using the regular solution model, determine the phase diagram
    of an isomorphous alloy that has
    been determined to have the solution parameters shown below. Use
    a computer (e.g., Python) to plot
    the free energy curves for strategic temperatures. The maximum
    temperature required for plotting
    your G-X curves should be easy to figure out. Figuring out what
    other strategic temperatures to use
    will require either playing around with the plots with respect to
    temperature (easy but potentially
    tedious) or using your thermodynamic prowess to directly solve
    for a useful intermediate temperature
    (harder and potentially much faster, but not required on this
    assignment). Then, produce at least
    eight different free energy plots at strategic temperatures. For
    this assignment, you can graphically
    determine the phase boundaries on the eight different plots and
    fill in the rest by hand to produce the
    phase diagram (i.e., connect the dots).

    \begin{align*}
      T_1^{\alpha\to L} &= \SI{1550}{\kelvin} \\
      T_2^{\alpha\to L} &= \SI{1200}{\kelvin} \\
      \Delta S_1^{0,\alpha\to L} &= \SI{8}{\joule\per\mole\per\kelvin} \\
      \Delta S_2^{0,\alpha\to L} &= \SI{11}{\joule\per\mole\per\kelvin} \\
      \omega_0^\alpha &= \SI{19500}{\joule\per\mole} \\
      \omega_0^L &= \SI{1500}{\joule\per\mole}
    \end{align*}

    \pagebreak

  \item Given the following equation for excess free energy of mixing
    of a solution, determine the listed
    changes in properties due to mixing 35 moles of A with 65 moles
    of B at 298 K and atmospheric
    pressure. I suggest you work symbolically until calculating the
    final answer of each part:

    \begin{align*}
      \Delta G_{\text{mix}}^{\text{xs}} &= a(1 - bT)(1 - cP)X_AX_B \\
      a &= \SI{10500}{\joule\per\mole} \\
      b &= \SI{3e-4}{\per\kelvin} \\
      c &= \SI{8e-10}{\per\pascal}
    \end{align*}

    \begin{enumerate}[(a)]
      \item  Total Gibbs free energy change of mixing.
      \item  Partial molar Gibbs free energy change of component A.
      \item  Partial molar Gibbs free energy change of component B.
      \item  Total entropy change of mixing.
      \item  Total enthalpy change of mixing.
      \item  Is this a regular solution? Concisely justify your answer.
    \end{enumerate}

\end{enumerate}

\pagebreak

\section*{Supporting code:}
\inputminted{julia}{./calculations/src/calculations.jl}

\end{document}
